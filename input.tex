\documentclass[dsc,male,12pt,a4paper]{ita}

\usepackage[brazil]{babel}
\usepackage{csquotes}
\usepackage{lipsum}

\usepackage[style=abnt,url=false,doi=false,isbn=false,maxbibnames=3,maxcitenames=3,natbib=true]{biblatex}

\addbibresource{./references.bib}

\title{Título do documento}
\author{Bruno}
\address{Minha Rua, n.ºXY \\ 12345-67, São José dos Campos - SP}
\keywords{}

\workcourse{Engenharia de Infraestrutura Aeronáutica}
\workarea{Infraestrutura Aeroportuária}

\advisor{female}{prof}{dr}{Minha Orientadora}{ITA}
\member{male}{prof}{dr}{Membro Interno}{memi}{ITA}
\member{female}{prof}{dr}{Membro Externo}{meme}{UFES}

\date{2024-08-31}

\begin{document}

\frontmatter
\maketitle
\makecip

\tableofcontents

\mainmatter
\chapter{Citações}
Uma citação ``normal'' \cite{silvaBIM4DNo2019}.

Segundo \textcite{dynamoDynamoBIM} é assim que cita no texto.

Posso citar várias fontes \cites{nrelEnergyPlus2024,ronzaniEstudoITAPara2020}.

\textcites{wenProgressTrendBIM2021,autodeskNavisworks3DModel2024} citam várias fontes também.

Assim como colocar informação adicional \cite[pre][post]{pauwelsIFCtoRDF2024}.

\chapter{Ambientes}

\begin{center}
	Esse texto é centralizado
\end{center}

\begin{itemize}
	\item Primeiro item
	\item Segundo item
	\item Terceiro item
\end{itemize}

\begin{itemize}
	\item Primeiro item
	      \begin{itemize}
		      \item Primeiro item aninhado
		      \item Segundo item aninhado
	      \end{itemize}
	\item Terceiro item
\end{itemize}

\begin{enumerate}
	\item Primeiro item numerado
	      \begin{itemize}
		      \item Item aninhado
		            \begin{enumerate}
			            \item Numerado duplamente aninhado
		            \end{enumerate}
	      \end{itemize}
	      \begin{enumerate}
		      \item Numerado aninhado
		            \begin{enumerate}
			            \item Ainda mais aninhado
		            \end{enumerate}
	      \end{enumerate}
\end{enumerate}


O parágrafo a seguir é uma citação
\begin{quotation}
	Essa é uma citação \cite{sacksBIMHandbookGuide2018}
\end{quotation}

E agora um quadro centralizado
\begin{center}
	\begin{tabular}{l|c|r}
		\hline
		Primeira linha & coluna central & última coluna \\
		Segunda linha  & outra central  & última        \\
		\hline
	\end{tabular}
\end{center}

\printbibliography
\end{document}
