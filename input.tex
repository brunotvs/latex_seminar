\documentclass[dsc,male,12pt,a4paper]{ita}

\usepackage[brazil]{babel}
\usepackage{csquotes}
\usepackage{lipsum}
\usepackage{amsmath}

\title{Título do documento}
\author{Bruno}

\workcourse{Engenharia de Infraestrutura Aeronáutica}
\workarea{Infraestrutura Aeroportuária}

\advisor{female}{prof}{dr}{Minha Orientadora}{ITA}
\coadvisor{male}{prof}{dr}{Meu Coorientador}{ITA}
\subscriber{male}{prof}{dr}{Nome do Pró-Reitor}{rec}

\date{2024-08-31}

\begin{document}

\frontmatter
\maketitle

% after
\listoffigures
\listoftables

\tableofcontents

\mainmatter
\chapter{Math}

Essa é uma equação no meio do texto $x = a + b$. Apesar de \$ funcionar, \textbackslash( é preferível, como em \(a = x - b\).


Podemos usar math env para fórmulas mais longas\par
\begin{math}
	F = m \times a
\end{math}

Ou, para equações numeradas:\par

\begin{equation}
	E = m \times c^2
\end{equation}


Para alinhar também podemos %\usepackage{amsmath}
% https://tex.stackexchange.com/questions/105635/align-multiline-equation-with-expression-after-equal-sign
\begin{align}
	\mathcal{L}^{-1}\left\{f(d)\right\} & = \mathcal{L}^{-1}\left\{f_1(\delta).f_2(\delta)\right\}                  \\
	                                    & = \exp(mt) \star \left\{\frac{l}{2\sqrt{\pi t^3}} \exp(-l^2/{4t})\right\} \\
	                                    & = F_1 * F_2
\end{align}

ou

\begin{equation}
	\begin{aligned}
		\mathcal{L}^{-1}\left\{f(d)\right\} & = \mathcal{L}^{-1}\left\{f_1(\delta).f_2(\delta)\right\}                 \\
		                                    & = \exp(mt) \star \left\{\frac{l}{2\sqrt{\pi t^3}} exp(-l^2/{4t})\right\} \\
		                                    & = F_1 * F_2
	\end{aligned}
\end{equation}

\end{document}
