\documentclass[dsc,male,12pt,a4paper,oneside]{ita}

\usepackage[brazil]{babel}
\usepackage{csquotes}
\usepackage{lipsum}

\usepackage[acronym]{glossaries}
\usepackage{glossaries-prefix}

\title{Título do documento}
\author{Bruno}

\workcourse{Engenharia de Infraestrutura Aeronáutica}
\workarea{Infraestrutura Aeroportuária}

\advisor{female}{prof}{dr}{Minha Orientadora}{ITA}
\coadvisor{male}{prof}{dr}{Meu Coorientador}{ITA}
\subscriber{male}{prof}{dr}{Nome do Pró-Reitor}{rec}

\date{2024-08-31}

\newacronym{EIA}{EIA}{Engenharia de Infraestrutura Aeronáutica}

% prefixo
\newacronym[prefix=an\space,prefixfirst=a\space]{RDF}{RDF}{Resource Description Framework}

\makeglossary

\begin{document}

\frontmatter
\maketitle

% after
\listoffigures
\listoftables

%print
\printglossary[type=\acronymtype, title={Abreviações}, nonumberlist]
\tableofcontents

\mainmatter
\chapter{Abreviações}

A primeira utilização possui o formato \gls{EIA}.


Utilizações seguintes são apresentadas como \gls{EIA}.


Mas eu posso pedir o completo usando \glsentrylong{EIA}.

Em inglês, podemos ter o problema de uma abreviação ter o prefixo ``a'' ou ``an''. Nesse caso, podemos estabelecer como queremos e usar \pgls{RDF} no primeiro uso ou \pgls{RDF} nos próximos.


\end{document}
